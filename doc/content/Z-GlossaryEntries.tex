% !TeX encoding=utf8
% !TeX spellcheck = de_CH_frami

%%% --- Acronym definitions
\IfDefined{newacronym}{%
% examples

% our used acronyms
\newacronym{zhawLabel}{ZHAW}{Zürcher Hochschule für Angewandte Wissenschaften}
\newacronym{ebsLabel}{EBS}{Einschreibe und Bewertungssystem der ZHAW}

\newacronym{macLabel}{MAC}{Media Access Control}


\newacronym{wepLabel}{WEP}{Wire Equivalency Protocol}
\newacronym{wpaLabel}{WPA}{Wi-Fi Protected Access}

\newacronym{apLabel}{AP}{Access Point}
\newacronym{ibssLabel}{IBSS}{Independent Basic Service Set}

\newacronym{pskLabel}{PSK}{pre-shared key}
\newacronym{pmkLabel}{PMK}{pairwise master key}
\newacronym{ptkLabel}{PTK}{pairwise transient key}
\newacronym{ivLabel}{IV}{Initialization Vector}


\newacronym{rtsLabel}{RTS}{Ready to send}
\newacronym{ctsLabel}{CTS}{Clear to send}

}

% special acronyms (used with glossary entry and see option)
\newglossaryentry{ssidLabel}{
	type=\acronymtype,
	name={SSID},
	description={Service Set ID},
	first={Service Set ID (SSID)\glsadd{glos:ssidLabel}},
	see=[Glossary:]{glos:ssidLabel}
}
\newglossaryentry{bssidLabel}{
	type=\acronymtype,
	name={BSSID},
	description={Basic Service Set ID},
	first={Basis Service Set ID (BSSID)\glsadd{glos:bssidLabel}},
	see=[Glossary:]{glos:bssidLabel}
}

%%% --- Symbol list entries

%\newglossaryentry{symb:Pi}{%
%  name=$\pi$,%
%  description={mathematical constant},%
%  sort=symbolpi, type=symbolslist%
%}


%%% --- Glossary entries
\newglossaryentry{glos:beaconLabel}{
	name=Beacon,
	plural={Beacons},
	description={Beacons sind Datenpakete, die vom \gls{apLabel} alle 200\,ms verschickt werden. Sie beinhalten Informationen zum Netzwerk, so dass neue Clients das Netzwerk erkennen und eine Verbindung (evt. mit Authentifizierung) anbieten können. Zudem kann ein verbundener Client über die regelmässigen Beacons feststellen, ob das Netzwerk ordnungsgemäss funktioniert.}
}

\newglossaryentry{glos:monitorModeLabel}{
	name={monitor mode},
	description={
		Der \textit{monitor mode} ist ein spezieller Modus einer Wireless Netzwerkkarte.
		Wird die Netzwerkkarte in diesen Modus versetzt, können alle Pakete eines Channels empfangen (und aufgezeichnet) werden.
		Der \textit{monitor mode} wird nicht von allen Netzwerkkarten unterstützt.
		Der Modus entspricht dem \textit{promiscuous mode} von kabelgebundenen Netzwerken.}
}

\newglossaryentry{glos:ssidLabel}{
	name={Service Set ID},
	description={Mit der \textit{Service Set ID (SSID)} wird ein gesamtes Netzwerk identifiziert. Die \gls{ssidLabel} entspricht dem sichtbaren Netzwerknamen und kann sich über mehrere \gls{apLabel}'s erstrecken.\footcite{Understanding_the_Network_Terms_SSID_BSSID_and_ESSID_Technical_Documentation_Support_2015-04-10}}
}

\newglossaryentry{glos:bssidLabel}{
	name={Basic Service Set ID},
	description={Mit der \textit{Basic Service Set ID (BSSID)} wird ein \gls{apLabel} identifiziert.
		Die \gls{bssidLabel} entspricht meist der \gls{macLabel}-Adresse des \gls{apLabel}'s und wird nicht manuell vergeben.\footcite{Understanding_the_Network_Terms_SSID_BSSID_and_ESSID_Technical_Documentation_Support_2015-04-10}}
}



%% Example for combined glossary and acronym: http://tex.stackexchange.com/questions/8946/how-to-combine-acronym-and-glossary
%\newglossaryentry{apig}{
%	name={API},
%	description={An Application Programming Interface (API) is a particular set
%		of rules and specifications that a software program can follow to access and
%		make use of the services and resources provided by another particular software
%		program that implements that API
%			%\footcite{...}
%	}
%}
%
%%%% define the acronym and use the see= option
%\newglossaryentry{api}{
%	type=\acronymtype,
%	name={API},
%	description={Application Programming Interface},
%	first={Application Programming Interface (API)\glsadd{apig}},
%	see=[Glossary:]{apig}
%}
%% use \gls{api}

% use it with \gls{glos:DVD}
% use plural with \glspl{thinClientLabel}