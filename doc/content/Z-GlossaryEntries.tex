% !TeX encoding=utf8
% !TeX spellcheck = de_CH_frami

%%% --- Acronym definitions
\IfDefined{newacronym}{%
% examples

% our used acronyms
% zhaw internal
\newacronym{zhawLabel}{ZHAW}{Zürcher Hochschule für Angewandte Wissenschaften}
\newacronym{ebsLabel}{EBS}{Einschreibe und Bewertungssystem der ZHAW}


% it general
\newacronym{macLabel}{MAC}{Media Access Control}
\newacronym{asciiLabel}{ASCII}{American Standard Code for Information Interchange}
\newacronym{httpsLabel}{HTTPS}{Hypertext Transfer Protocol Secure}

\newacronym{ieeeLabel}{IEEE}{Institute of Electrical and Electronics Engineers}
\newacronym{usbLabel}{USB}{Universal Serial Bus}
\newacronym{nfcLabel}{NFC}{Near Field Communication}
\newacronym{cpuLabel}{CPU}{Central Processing Unit}
\newacronym{gpuLabel}{GPU}{Graphics Processing Unit}

\newacronym{tkipLabel}{TKIP}{Temporal Key Integrity Protocol}
\newacronym{aesLabel}{AES}{Advanced Encryption Standard}


%network
\newacronym{arpLabel}{ARP}{Address Resolution Protocol}
\newacronym{rtsLabel}{RTS}{Ready to send}
\newacronym{ctsLabel}{CTS}{Clear to send}

% wireless basics
\newacronym{wlanLabel}{WLAN}{Wireless Local Area Network}

\newacronym{wepLabel}{WEP}{Wire Equivalency Protocol}
\newacronym{wpaLabel}{WPA}{Wi-Fi Protected Access}
\newacronym{wpa2Label}{WPA2}{Wi-Fi Protected Access 2}
\newacronym{wpsLabel}{WPS}{Wi-Fi Protected Setup}

\newacronym{apLabel}{AP}{Access Point}
\newacronym{ibssLabel}{IBSS}{Independent Basic Service Set}

\newacronym{pskLabel}{PSK}{pre-shared key}
\newacronym{pmkLabel}{PMK}{pairwise master key}
\newacronym{ptkLabel}{PTK}{pairwise transient key}

% wep
\newacronym{ivLabel}{IV}{initialization vector}

% wpa
\newacronym{micLabel}{MIC}{message integrity check}
\newacronym{hmacshaLabel}{HMAC-SHA1}{keyed-hash message authentication code secure hash algorithm}




}

%% special acronyms (used with glossary entry and see option)
%\newglossaryentry{ssidLabel}{
%	type=\acronymtype,
%	name={SSID},
%	description={Service Set ID},
%	first={Service Set ID (SSID)\glsadd{glos:ssidLabel}},
%	see=[Glossary:]{glos:ssidLabel}
%}
%\newglossaryentry{bssidLabel}{
%	type=\acronymtype,
%	name={BSSID},
%	description={Basic Service Set ID},
%	first={Basis Service Set ID (BSSID)\glsadd{glos:bssidLabel}},
%	see=[Glossary:]{glos:bssidLabel}
%}

%%% --- Symbol list entries

%\newglossaryentry{symb:Pi}{%
%  name=$\pi$,%
%  description={mathematical constant},%
%  sort=symbolpi, type=symbolslist%
%}


%%% --- Glossary entries
\newglossaryentry{glos:beaconLabel}{
	name=Beacon,
	plural={Beacons},
	description={Beacons sind Datenpakete, die vom \gls{apLabel} alle 200\,ms verschickt werden. Sie beinhalten Informationen zum Netzwerk, so dass neue Clients das Netzwerk erkennen und eine Verbindung (evt. mit Authentifizierung) anbieten können. Zudem kann ein verbundener Client über die regelmässigen Beacons feststellen, ob das Netzwerk ordnungsgemäss funktioniert.}
}

\newglossaryentry{glos:monitorModeLabel}{
	name={monitor mode},
	description={
		Der \textit{monitor mode} ist ein spezieller Modus einer Wireless Netzwerkkarte.
		Wird die Netzwerkkarte in diesen Modus versetzt, können alle Pakete eines Channels empfangen (und aufgezeichnet) werden.
		Der \textit{monitor mode} wird nicht von allen Netzwerkkarten unterstützt.
		Der Modus entspricht dem \textit{promiscuous mode} von kabelgebundenen Netzwerken.}
}

\newglossaryentry{glos:onPremiseLabel}{name=on-premise,
	description={On-premise Software bezeichnet jegliche Ausführung von Software die vorwiegend auf dem Endgerät selbst läuft. Alternativ existiert das \Gls{glos:cloudLabel} Modell, bei dem der Grossteil der Software auf einem Server ausgeführt wird.}
}
\newglossaryentry{glos:cloudLabel}{
	name=Cloud,
	description={\textit{Cloud} bietet dem Benutzer Software oder Hardware zur Benutzung an, ohne das jener im Detail weiss, auf was für Infrastruktur er arbeitet. Meist stehen Komponenten im Internet bereit, so dass weltweiter Zugriff ermöglicht wird.}
}

\newglossaryentry{glos:bruteforceLabel}{name=Bruteforce,
	description={Bruteforce (engl. \textit{brute force}, \textit{rohe Gewalt}) definiert das ausprobieren möglicher Kombinationen. Bei der Kryptografie ist dies eine sichere, jedoch ineffiziente Lösung um ein Passwort herauszufinden.}
}



%Glossary entries used also as acronyms
\newglossaryentry{ssidLabel}{
	name={SSID},
	description={Mit der \textit{Service Set ID (SSID)} wird ein gesamtes Netzwerk identifiziert. Die \textit{SSID} entspricht dem sichtbaren Netzwerknamen und kann sich über mehrere \gls{apLabel}'s erstrecken.\footcite{Understanding_the_Network_Terms_SSID_BSSID_and_ESSID_Technical_Documentation_Support_2015-04-10}},
	first={Service Set ID (SSID)},
	long={Service Set ID}
}

\newglossaryentry{bssidLabel}{
	name={BSSID},
	description={Mit der \textit{Basis Service Set ID (BSSID)} wird ein \gls{apLabel} identifiziert.
		Die \textit{BSSID} entspricht meist der \gls{macLabel}-Adresse des \gls{apLabel}'s und wird nicht manuell vergeben.\footcite{Understanding_the_Network_Terms_SSID_BSSID_and_ESSID_Technical_Documentation_Support_2015-04-10}},
	first={Basis Service Set ID (BSSID)},
	long={Basis Service Set ID}
}

%\newglossaryentry{glos:ssidLabel}{
%	name={Service Set ID},
%	description={Mit der \textit{Service Set ID (SSID)} wird ein gesamtes Netzwerk identifiziert. Die \gls{ssidLabel} entspricht dem sichtbaren Netzwerknamen und kann sich über mehrere \gls{apLabel}'s erstrecken.\footcite{Understanding_the_Network_Terms_SSID_BSSID_and_ESSID_Technical_Documentation_Support_2015-04-10}}
%}

%\newglossaryentry{glos:bssidLabel}{
%	name={Basic Service Set ID},
%	description={Mit der \textit{Basic Service Set ID (BSSID)} wird ein \gls{apLabel} identifiziert.
%		Die \gls{bssidLabel} entspricht meist der \gls{macLabel}-Adresse des \gls{apLabel}'s und wird nicht manuell vergeben.\footcite{Understanding_the_Network_Terms_SSID_BSSID_and_ESSID_Technical_Documentation_Support_2015-04-10}}
%}


% Example for combined glossary and acronym: http://tex.stackexchange.com/questions/8946/how-to-combine-acronym-and-glossary
%\newglossaryentry{api11}
%{
%	name={API},
%	description={An Application Programming Interface (API) is a particular set
%		of rules and specifications that a software program can follow to access and make use of the services and resources provided by another particular software program that implements that API},
%	first={Application Programming Interface (API)},
%	long={Application Programming Interface}
%}

%% use \gls{api}

% use it with \gls{glos:DVD}
% use plural with \glspl{thinClientLabel}