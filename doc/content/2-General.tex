\chapter{Allgemein}

\section{Modus}
Eine drahtlose Umgebung kann in zwei verschiedenen Modis betrieben werden.
Meist funktioniert die Kommunikation via einen \gls{apLabel}, über der alles läuft. Der \gls{apLabel} leitet die Pakete weitere und kontrolliert die Verbindung.
Nebst der \gls{apLabel}-Variante gibt es noch ad-hoc oder \gls{ibssLabel} Modus. Auch hier gibt es einen \gls{apLabel}, aber die Pakete müssen nicht zwingend über jenen geleitet werden, sondern die Stationen können direkt miteinander kommunizieren.\footcite[][38]{WrightCache201503}

\section{Paket Typ}
Die Pakete können in drei Kategorien unterschieden werden:
\begin{itemize}
	\item Data (Daten): Mit \textit{Data}-Paketen werden alle Informationen bezeichnet, die grundsätzlich übertragen werden müssen.
	\item Management (Verwaltung): \textit{Management} Pakete bezeichnet alle Pakete, welche Informationen zum Netzwerk bereitstellen (z.B. Beacons und Deauthentication Pakete)
	\item Control (Kontrolle): \textit{Control}-Pakete steuern die Kommunikation. Sie regeln z.B. wer wann senden darf (z.B.: \gls{rtsLabel} und \gls{ctsLabel}).
\end{itemize}

\section{Adresse}
Jedes Paket beinhaltet, nebst Quell- und Zieladresse, auch eine \gls{bssidLabel}, welche den \gls{apLabel} identifiziert (meist entspricht sie der \gls{macLabel}-Adresse des \gls{apLabel}).


\section{Verschlüsselungstechnik}
Es gibt zwei Arten auf welche die Kommunikation verschlüsselt werden kann: \gls{wepLabel} (\chapref{ch:wep}) und \gls{wpaLabel} (\chapref{ch:wpa}).
Beide Arten unterstützen die \gls{pskLabel} Methode, bei der für alle Nutzer der gleiche Schlüssel für Authentifizierung und Verschlüsselung verwendet wird.
\gls{wpaLabel} bietet alternativ auch eine \textit{enterprise} Methode an, bei der ein Authentifizierung-Server jeden Nutzer einzeln authentifiziert und für jede Sitzung einen neuen \gls{pskLabel} Schlüssel generiert. Dieses \textit{enterprise} Protokoll wird oft in Firmen angewendet, für den privaten Gebrauch lohnt sich der Einrichtungsaufwand und Betrieb eines eigenen Servers nicht.

\section{Attacken}

\subsection{Sicheheitslevel}
Netzwerke können grob in drei Sicherheitslevel eingestuft werden:\footcite[][115]{WrightCache201503}
\begin{itemize}
	\item \textbf{Geringe Sicherheit:} Von geringer Sicherheit spricht man, wenn man ohne oder mit sehr geringen Aufwand Sicherheitsmassnahmen umgehen kann.
	\item \textbf{Mittlere Sicherheit:} Mittlerer Sicherheit, kann mit mässigem Einsatz von Ressourcen (Zeit und Wissen) umgangen werden (z.B. \gls{wepLabel}-Schlüssel oder ein schwacher \gls{wpaLabel}-Schlüssel).
	\item \textbf{Hohe Sicherheit:} Hohe Sicherheitsmassnahmen können kaum oder nur mit extremen Aufwand umgangen werden.
\end{itemize}

\subsection{Angriffsarten}
%todo aktiv/passiv

\subsection{Angriffsziele}
Ziel eines Angriffes kann das Abhören (evt. inkl. Manipulation) oder der vollständigen Netzwerkzugriff (via \gls{pskLabel}) sein.
Meist wird der vollständige Netzwerkzugriff angestrebt, da dann auch alle Kommunikation abgehört werden kann.

\section{Test Umgebung (Soft- und Hardware)}
\label{sec:testEnvroiment}
\subsection{System}
Für die praktischen Beispiele wurde ein Apple \textit{Mac Book Air}(early 2014) mit einem 1.4GHz Intel Core i5 Prozessor und 8 GB RAM verwendet. Als Grafikkarte wurde die eingebaute, standard \textit{Intel HD Graphics 5000} mit 1536 MB Shared Memory, verwendet
Als Betriebssystem wurde die OS X Version 10.10.2 \textit{"`Yosemite"'} verwendet.

Als Netzwerkkarte wurde die eingebaute \textit{AirPort Extreme} mit der Firmware \textit{Broadcom BCM43xx 1.0 (7.15.159.13.12)} verwendet.

Falls nichts anderes erwähnt wird, wurde immer soeben beschriebene Hardware verwendet.
