\chapter{Wire Equivalency Protocol (WEP)}
\label{ch:wep}

\section{Allgemein}
Als erstes, weit verbreitetes Sicherheitsprotokoll hat sich \acrfull{wepLabel} durchgesetzt.
Trotzdem, dass \gls{wepLabel} grosse Schwachstellen und Sicherheitslücken aufweist und diese seit einiger Zeit auch bekannt ist, wird \gls{wepLabel} immer noch angewendet (oft von alten Geräten oder auch neue, die aus Kompatibilitäts-Gründen mehrere Verbindungen gewähren).

Der Schlüssel ist theoretisch 64 oder 128 Bits lang.
Da davon jedoch 24 Bits für den \gls{ivLabel} verwendet werde, und der unverschlüsselt übermittelt wird, kürzt sich die Schlüssellänge erheblich.
Sprich die effektive Länge des Schlüssel ist 40 oder 104 Bits.
%Die verkürzte Schlüssellänge ist jedoch nicht das Hauptproblem von \gls{wepLabel}.

\section{Angriffe (WEP)}
Nebst fehlerhaften \gls{wepLabel} Implementationen konkreter Hersteller, können allgemeine \gls{wepLabel}-Angriffe in zwei Kategorien eingeteilt werden.\footcite[][126f.]{WrightCache201503}
Nämlich in Netzwerke mit Clients (und entsprechendem Datenverkehr) und solche ohne aktive Clients.

\subsection{Angriff mit einem Opfer Client (WEP)}
Der Angriff auf ein Netzwerk mit Clients ist einfacher, kann jedoch mehr Geduld beanspruchen.

Ein durchgeführtes Beispiel ist im \secref{sec:wepAttack} dokumentiert.





\subsection{Angriff ohne einem Opfer Client (WEP)}


\section{Konkreter Angriff}
\label{sec:wepAttack}

\subsection{Setup}
\subsection{Hardware}
Nebst der in \secref{sec:testEnvroiment} beschriebene Hardware wurde ein ältere Swisscom Router der "`Netopia  3357NWG"'\footcite{Netopia_3347_57nwg_de_2015-04-06} verwendet. Der Router wurde mit einem 10-stelligen hexadezimalen Passwort eingerichtet.

\subsection{Software}
OS X stellt, innerhalb von Xcode\footcite{Xcode_Apple_Developer_2015-04-06}, bereits nützliche Wireless Analyse Tools zur Verfügung.
Zudem kann via "`Homebrew"'\footcite{Homebrew__The_missing_package_manager_for_OS_X_2015-04-06} die "`aircrack-ng"'\footcite{Aircrack-ng_2015-04-06} Programm Sammlung installiert werden.

\begin{framed}
	\textbf{Bemerkung:} Obwohl das "`KisMac2"'\footcite{IGR_Software_KisMac2_2015-04-06} viel erwähnt wird, hat sich dieses Programm während dieser Arbeit nicht bewährt. Es stürzte vermehrt ab. Darum die Empfehlung mit dem von Xcode bereitgestellten "`airport"' Programm zu arbeiten.
\end{framed}

\subsection{Beispiel}
Mit dem "`Wireless Diagnostics"' Tool von Apple kann via \textit{Window} $\rightarrow$ das \textit{Scan} und das \textit{Sniffer} Programm geöffnet werden.
Mit dem \textit{Scan} Programm können verfügbare Netze und deren Channels gefunden werden.

\begin{figure}[H]
	\centering
	\includegraphics[width=1.0\textwidth]{images/wep/scan.png}
	\caption{Scanner -- OS X Wireless Diagnostics}
\end{figure}

Mit dem \textit{Sniffer} können benötigte Daten auf einem konkreten Netz aufgezeichnet werden. Es werden ca. 5'000 \gls{ivLabel}'s benötigt um den Schlüssel zu berechnen. (Diese Aufzeichnung (capture) dauert unterschiedlich lange, je nach Nutzung des Netzes. Wird das Netz viel gebraucht, geht es rascher.)

\begin{figure}[H]
	\centering
	\includegraphics[width=0.8\textwidth]{images/wep/sniffer.png}
	\caption{Sniffer -- OS X Wireless Diagnostics}
\end{figure}

%\begin{wrapfigure}{r}{0.6\textwidth}
%	\includegraphics[width=1.0\linewidth]{images/wep/sniffer.png}
%	\caption{OS X Wireless Diagnostics: Sniffer}
%\end{wrapfigure}

Der Sniffer kann auch direkt über die Kommandozeile ausgeführt werden:
\begin{lstlisting}[style=lstStyleFramed]
sudo ln -s /System/Library/PrivateFrameworks/Apple80211.framework/Versions/Current/Resources/airport /usr/sbin/airport
sudo airport en0 sniff <CHANNEL>
\end{lstlisting}

Anschliessend können die Daten, vom \textit{Sniffer} mit dem \textit{aircrack-ng} ausgewertet werden.
\begin{lstlisting}[style=lstStyleFramed]
aircrack-ng -b 00:0f:cc:b0:1a:c8 /private/tmp/airportSniff*.cap
\end{lstlisting}


