\chapter{Wi-Fi Protected Access (WPA)}
\label{ch:wpa}


\section{WPA Einführung}
\todo{pre-shared (incl. WPS) or enterprise}

Das \gls{wpaLabel} Protokoll wurde als notwendiger Nachfolger des \gls{wepLabel} Protokolls definiert.
Auch war der weiterführender, sehr umfangreichen Standard \textit{\gls{ieeeLabel} 802.11i} in Arbeit, war aber zum Zeitpunkt als \gls{wepLabel} geknackt wurde noch nicht abgeschlossen. So wurde Anfangs 2013 die ersten zertifizierten \gls{wpaLabel}-fähigen \gls{wlanLabel}-Geräte auf den Markt gebracht.
Seit September 2004 gibt es zertifizierte \gls{wpa2Label} Netzwerkgeräte. Der \gls{wpa2Label} entspricht nun dem \textit{\gls{ieeeLabel} 802.11i} Standard.\footcite{Wi-Fi_Protected_Access__Wikipedia_2015-04-10}\footcite{WPA2__Wikipedia_2015-04-10}

Der \gls{wpa2Label} Standard gilt bis heute als sicher.
Ein erfolgreicher Angriff muss via Bruteforce und Wörterbuch Angriff geschehen, was relativ unmöglich ist, vorausgesetzt es wird ein sicherer \gls{pskLabel} verwendet (ab 16 zufälligen alphanumerischen Zeichen).

\subsection{Schlüsselaustausch}
% going on
Voraussetzung für den gewöhnlichen Schlüsselaustausch ist der Besitz des \gls{pskLabel} (entspricht dem \textit{passphrase}).
Dieser Schlüssel kann aus 8 bis 63 druckbaren ASCII Zeichen bestehen.

Bei einer Verbindung zum Netzwerk wird beidseitig aus dem \gls{pskLabel} und der \gls{ssidLabel} der \gls{pmkLabel} berechnet.
Der \gls{apLabel} überträgt dem Client eine zufällige Zahl \textit{A-nonce}, welche der Client mit einer weiteren zufälligen Zahl ergänzt \textit{B-nonce}.
Der \gls{apLabel} berechnet aus den zufälligen Zahlen und den beiden \gls{macLabel}-Adressen einen temporären \gls{ptkLabel}, der pro Session neu generiert wird. Der \gls{ptkLabel} wird zudem auch während eines Session periodisch neu ausgewechselt.\footcite[][40f.]{WrightCache201503}

\subsection{Authentifizierungs-Varianten}
\gls{wpaLabel} bietet nebst der \gls{pskLabel} Methode, auch eine \textit{enterprise} Methode an, bei der ein Authentifizierung-Server jeden Nutzer einzeln authentifiziert und für jede Sitzung einen neuen \gls{pmkLabel} Schlüssel generiert.
Dieses \textit{enterprise} Protokoll wird oft in Firmen angewendet; lohnt sich aber für den privaten Gebrauch nicht, da der Einrichtungsaufwand und Betrieb eines eigenen Servers zu hoch ist.
Deshalb wird in dieser Arbeit auf die \gls{pskLabel} Methode eingegangen.

\subsubsection{WPS Standard}
Der \gls{wpsLabel} Standard entspricht nicht einer eigener Zugriffs Methode, sondern soll das Einrichten von sicheren \gls{pskLabel} beim Client erleichtern.

Dazu gibt es folgende vier Methoden:\footcite{Wi-Fi_Protected_Setup_-_Wikipedia_the_free_encyclopedia_2015-04-10}
\begin{itemize}
	\item \textbf{PIN:} Beim \gls{apLabel} muss ein 8-stelliger PIN des Clients eingegeben werden oder umgekehrt.
	\item \textbf{Push button:} Per Kopfdruck am \gls{apLabel} wird eine zweiminütige Beitrittsphase des Netzwerkes gestartet.
	\item \textit{\gls{nfcLabel}:} Die Daten werden via \gls{nfcLabel} ausgetauscht (nur innerhalb kurzer Distanz möglich).
	\item \textit{\gls{usbLabel}:} Die Netzinformationen werden via externen \gls{usbLabel} ausgetauscht.
\end{itemize}

Die PIN Methode gilt als besonders unsicher, da lediglich eine 7-stellige Zahl erraten werden muss (die letzte entfällt, da sie einer Checksumme entspricht).
Als Gegenmassnahme kann eine automatische zeitliche Sperre nach fehlgeschlagenen Versuchen eingestellt werden.\footcite{viehboeck_wps_2015-04-10}

\textbf{Push button} und \textit{\gls{nfcLabel}} sind anfällig auf ungewollte Nutzer, wenn der \gls{apLabel} an einem unbewachten Ort steht und sich so jeder authentifizieren kann.



\section{Angriffe}
%p148    \footcite[][148f.]{WrightCache201503}:

