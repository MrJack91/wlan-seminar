\chapter{Beschreibung der Aufgabe}

\section{Einführung}
\subsection{Zielpublikum}
Trotzdem im Glossar einige Fachbegriffe erklärt werden, wird für das Verstehen dieser Arbeit technisches Verständnis vorausgesetzt.

Eine konsequente Übersetzung englischer Begriffe würde keinen Beitrag bezüglich Verständlichkeit leisten und wurde daher weggelassen.

\subsection{Hinweis zu verwendetem E-Book}
Viele Teile der Arbeit stützen sich auf dem E-Book "`Hacking Exposed Wireless"'.\footcite{WrightCache201503}
Dadurch, dass eine E-Book Version verwendet wird, entsprechen die Seitenzahlen nicht deren der ausgedruckten Variante.
Der Inhalt entspricht jedoch dem Selben.

\subsection{Ressourcen}
Weitere Dateien wie der Latex Source dieses Dokuments ist \url{https://github.com/MrJack91/wlan_seminar} verfügbar.
Verwendete Aufzeichnungen (captures) sind unter \url{https://github.com/MrJack91/wlan_seminar/tree/master/practise/data} abgelegt.

\section{Aufgabenstellung}
% gemäss EBS
Folgende Aufgabenstellung wurde direkt aus dem \gls{ebsLabel} entnommen.

\subsection{Ausgangslage}
Im Geschäftsbereich, wie auch in privaten Haushalten, wird Wireless sehr oft genutzt.
Wireless bietet dem Nutzer viele Vorteile und Freiheiten. Nebst einer wirelessfähigen Netzwerkkarte, die meist im Endgerät schon eingebaut ist, wird keine zusätzliche Hardware pro Benutzer (resp. Gerät) benötigt.

Da die Verbindung via Funk ist, eignet sich Wireless auch für mobile Geräte.

\subsection{Ziel der Arbeit}
In dieser Arbeit soll eine Übersicht über momentan verbreiteten (sprich nicht nur den aktuellen) Wireless Security Standards und deren Schwachstellen aufgezeigt werden.
Mögliche Angriffe auf Sicherheitsrisiken sollen exemplarisch, im geschützten Bereich, ausgetestet werden.

\subsection{Aufgabenstellung}
Im Rahmen der Arbeit sollen folgende Aufgaben recherchiert und durchgeführt werden:
\begin{itemize}
	\item Übersicht über aktuell verbreiteten Wireless Security Standards
	\item Bekannte Schwachstellen je Standard, inkl. Beschreibung und Durchführung von Angriffen
	\item Definition "`sichere/sicherste Wireless-Konfiguration"' aus heutiger Sicht
\end{itemize}

\subsection{Erwartete Resultate}
Das resultierende Dokument soll eine Übersicht der momentan verbreiteten Wireless Standards und deren Sicherheitsrisiken aufzeigen.
Zudem sollen mögliche Angriffe beschrieben und durchgeführt werden.

\subsection{Eingrenzung}
Es wird auf Wireless im Allgemeinen eingegangen (\chapref{ch:general}).
Zudem wird Fokus auf den \gls{wepLabel}-Standard (\chapref{ch:wep}), so wie der \gls{wpaLabel}-Standard (\chapref{ch:wpa}) gelegt. Beide Standards werden anhand eines konkreten Beispiels angegriffen.

\subsection{Abgrenzung}
Die aufgeführten Kapitel sind nicht vollständig, sondern dienen dazu einen fundierten Einblick ins Thema "`Wireless Security Standard"' und möglicher Angriffsszenarien zu gewinnen.

