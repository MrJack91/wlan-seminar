\chapter{Beschreibung der Aufgabe}

\section{Aufgabenstellung}
% gemäss EBS
Folgende Aufgabenstellung wurde direkt aus dem \gls{ebsLabel} entnommen.

\subsection{Ausgangslage}
Im Geschäftsbereich, wie auch in privaten Haushalten wird Wireless sehr oft genutzt.
Wireless bietet dem Nutzer viele Vorteile und Freiheiten. Nebst einer wirelessfähigen Netzwerkkarte, die meist im Endgerät schon eingebaut ist, wird keine zusätzliche Hardware pro Benutzer (resp. Gerät) benötigt.

Da die Verbindung via Funk ist, eignet sich Wireless für mobile Geräte.

\subsection{Ziel der Arbeit}
In dieser Arbeit soll eine Übersicht über momentan verbreiteten (sprich nicht nur den aktuellen) Wireless Security Standards und deren Schwachstellen aufgezeigt werden.
Mögliche Angriffe auf Sicherheitsrisiken sollen exemplarisch, im geschützten Bereich, ausgetestet werden.

\subsection{Aufgabenstellung}
Im Rahmen der Arbeit sollen folgende Aufgaben recherchiert und durchgeführt werden:
\begin{itemize}
	\item Übersicht über aktuell verbreiteten Wireless Security Standards
	\item Bekannte Schwachstellen je Standard, inkl. Beschreibung und Durchführung von Angriffen
	\item Definition "sichere/sicherste Wireless Konfiguration" aus heutiger Sicht
\end{itemize}

\subsection{Erwartete Resultate}
Das resultierende Dokument soll eine Übersicht der momentan verbreiteten Wireless Standards und deren Sicherheits-Risiken aufzeigen.
Zudem sollen mögliche Angriffe beschrieben und durchgeführt werden.


\subsection{Eingrenzung}
%todo 

\subsection{Abgrenzung}
%todo 

